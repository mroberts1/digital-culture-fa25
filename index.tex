% Options for packages loaded elsewhere
% Options for packages loaded elsewhere
\PassOptionsToPackage{unicode}{hyperref}
\PassOptionsToPackage{hyphens}{url}
\PassOptionsToPackage{dvipsnames,svgnames,x11names}{xcolor}
%
\documentclass[
  letterpaper,
  DIV=11,
  numbers=noendperiod]{scrartcl}
\usepackage{xcolor}
\usepackage{amsmath,amssymb}
\setcounter{secnumdepth}{-\maxdimen} % remove section numbering
\usepackage{iftex}
\ifPDFTeX
  \usepackage[T1]{fontenc}
  \usepackage[utf8]{inputenc}
  \usepackage{textcomp} % provide euro and other symbols
\else % if luatex or xetex
  \usepackage{unicode-math} % this also loads fontspec
  \defaultfontfeatures{Scale=MatchLowercase}
  \defaultfontfeatures[\rmfamily]{Ligatures=TeX,Scale=1}
\fi
\usepackage{lmodern}
\ifPDFTeX\else
  % xetex/luatex font selection
\fi
% Use upquote if available, for straight quotes in verbatim environments
\IfFileExists{upquote.sty}{\usepackage{upquote}}{}
\IfFileExists{microtype.sty}{% use microtype if available
  \usepackage[]{microtype}
  \UseMicrotypeSet[protrusion]{basicmath} % disable protrusion for tt fonts
}{}
\makeatletter
\@ifundefined{KOMAClassName}{% if non-KOMA class
  \IfFileExists{parskip.sty}{%
    \usepackage{parskip}
  }{% else
    \setlength{\parindent}{0pt}
    \setlength{\parskip}{6pt plus 2pt minus 1pt}}
}{% if KOMA class
  \KOMAoptions{parskip=half}}
\makeatother
% Make \paragraph and \subparagraph free-standing
\makeatletter
\ifx\paragraph\undefined\else
  \let\oldparagraph\paragraph
  \renewcommand{\paragraph}{
    \@ifstar
      \xxxParagraphStar
      \xxxParagraphNoStar
  }
  \newcommand{\xxxParagraphStar}[1]{\oldparagraph*{#1}\mbox{}}
  \newcommand{\xxxParagraphNoStar}[1]{\oldparagraph{#1}\mbox{}}
\fi
\ifx\subparagraph\undefined\else
  \let\oldsubparagraph\subparagraph
  \renewcommand{\subparagraph}{
    \@ifstar
      \xxxSubParagraphStar
      \xxxSubParagraphNoStar
  }
  \newcommand{\xxxSubParagraphStar}[1]{\oldsubparagraph*{#1}\mbox{}}
  \newcommand{\xxxSubParagraphNoStar}[1]{\oldsubparagraph{#1}\mbox{}}
\fi
\makeatother


\usepackage{longtable,booktabs,array}
\usepackage{calc} % for calculating minipage widths
% Correct order of tables after \paragraph or \subparagraph
\usepackage{etoolbox}
\makeatletter
\patchcmd\longtable{\par}{\if@noskipsec\mbox{}\fi\par}{}{}
\makeatother
% Allow footnotes in longtable head/foot
\IfFileExists{footnotehyper.sty}{\usepackage{footnotehyper}}{\usepackage{footnote}}
\makesavenoteenv{longtable}
\usepackage{graphicx}
\makeatletter
\newsavebox\pandoc@box
\newcommand*\pandocbounded[1]{% scales image to fit in text height/width
  \sbox\pandoc@box{#1}%
  \Gscale@div\@tempa{\textheight}{\dimexpr\ht\pandoc@box+\dp\pandoc@box\relax}%
  \Gscale@div\@tempb{\linewidth}{\wd\pandoc@box}%
  \ifdim\@tempb\p@<\@tempa\p@\let\@tempa\@tempb\fi% select the smaller of both
  \ifdim\@tempa\p@<\p@\scalebox{\@tempa}{\usebox\pandoc@box}%
  \else\usebox{\pandoc@box}%
  \fi%
}
% Set default figure placement to htbp
\def\fps@figure{htbp}
\makeatother





\setlength{\emergencystretch}{3em} % prevent overfull lines

\providecommand{\tightlist}{%
  \setlength{\itemsep}{0pt}\setlength{\parskip}{0pt}}



 


% load packages
\usepackage{geometry}
\usepackage{xcolor}
\usepackage{eso-pic}
\usepackage{fancyhdr}
\usepackage{sectsty}
\usepackage{fontspec}
\usepackage{titlesec}

%% Set page size with a wider right margin
\geometry{a4paper, total={170mm,257mm}, left=20mm, top=20mm, bottom=20mm, right=50mm}

%% Let's define some colours
\definecolor{light}{HTML}{E6E6FA}
\definecolor{highlight}{HTML}{800080}
\definecolor{dark}{HTML}{330033}

%% Let's add the border on the right hand side 
% \AddToShipoutPicture{% 
%     \AtPageLowerLeft{% 
%         \put(\LenToUnit{\dimexpr\paperwidth-3cm},0){% 
%             \color{light}\rule{3cm}{\LenToUnit\paperheight}%
%           }%
%      }%
%      % logo
%     \AtPageLowerLeft{% start the bar at the bottom right of the page
%         \put(\LenToUnit{\dimexpr\paperwidth-2.25cm},27.2cm){% move it to the top right
%             \color{light}\includegraphics[width=1.5cm]{_extensions/nrennie/PrettyPDF/logo.png}
%           }%
%      }%
% }

%% Style the page number
\fancypagestyle{mystyle}{
  \fancyhf{}
  \renewcommand\headrulewidth{0pt}
  \fancyfoot[R]{\thepage}
  \fancyfootoffset{3.5cm}
}
\setlength{\footskip}{20pt}

%% style the chapter/section fonts
\chapterfont{\color{dark}\fontsize{20}{16.8}\selectfont}
\sectionfont{\color{dark}\fontsize{20}{16.8}\selectfont}
\subsectionfont{\color{dark}\fontsize{14}{16.8}\selectfont}
\titleformat{\subsection}
  {\sffamily\Large\bfseries}{\thesection}{1em}{}[{\titlerule[0.8pt]}]
  
% left align title
\makeatletter
\renewcommand{\maketitle}{\bgroup\setlength{\parindent}{0pt}
\begin{flushleft}
  {\sffamily\huge\textbf{\MakeUppercase{\@title}}} \vspace{0.3cm} \newline
  {\Large {\@subtitle}} \newline
  \@author
\end{flushleft}\egroup
}
\makeatother

%% Use some custom fonts
\setsansfont{Ubuntu}[
    Path=_extensions/nrennie/PrettyPDF/Ubuntu/,
    Scale=0.9,
    Extension = .ttf,
    UprightFont=*-Regular,
    BoldFont=*-Bold,
    ItalicFont=*-Italic,
    ]

\setmainfont{Ubuntu}[
    Path=_extensions/nrennie/PrettyPDF/Ubuntu/,
    Scale=0.9,
    Extension = .ttf,
    UprightFont=*-Regular,
    BoldFont=*-Bold,
    ItalicFont=*-Italic,
    ]
\KOMAoption{captions}{tableheading}
\makeatletter
\@ifpackageloaded{caption}{}{\usepackage{caption}}
\AtBeginDocument{%
\ifdefined\contentsname
  \renewcommand*\contentsname{Table of contents}
\else
  \newcommand\contentsname{Table of contents}
\fi
\ifdefined\listfigurename
  \renewcommand*\listfigurename{List of Figures}
\else
  \newcommand\listfigurename{List of Figures}
\fi
\ifdefined\listtablename
  \renewcommand*\listtablename{List of Tables}
\else
  \newcommand\listtablename{List of Tables}
\fi
\ifdefined\figurename
  \renewcommand*\figurename{Figure}
\else
  \newcommand\figurename{Figure}
\fi
\ifdefined\tablename
  \renewcommand*\tablename{Table}
\else
  \newcommand\tablename{Table}
\fi
}
\@ifpackageloaded{float}{}{\usepackage{float}}
\floatstyle{ruled}
\@ifundefined{c@chapter}{\newfloat{codelisting}{h}{lop}}{\newfloat{codelisting}{h}{lop}[chapter]}
\floatname{codelisting}{Listing}
\newcommand*\listoflistings{\listof{codelisting}{List of Listings}}
\makeatother
\makeatletter
\makeatother
\makeatletter
\@ifpackageloaded{caption}{}{\usepackage{caption}}
\@ifpackageloaded{subcaption}{}{\usepackage{subcaption}}
\makeatother
\makeatletter
\@ifpackageloaded{tcolorbox}{}{\usepackage[skins,breakable]{tcolorbox}}
\makeatother
\makeatletter
\@ifundefined{shadecolor}{\definecolor{shadecolor}{rgb}{.97, .97, .97}}{}
\makeatother
\makeatletter
\@ifundefined{codebgcolor}{\definecolor{codebgcolor}{named}{light}}{}
\makeatother
\makeatletter
\ifdefined\Shaded\renewenvironment{Shaded}{\begin{tcolorbox}[boxrule=0pt, colback={codebgcolor}, breakable, frame hidden, enhanced, sharp corners]}{\end{tcolorbox}}\fi
\makeatother
\usepackage{bookmark}
\IfFileExists{xurl.sty}{\usepackage{xurl}}{} % add URL line breaks if available
\urlstyle{same}
\hypersetup{
  pdftitle={VM303-01 Studies in Digital Media \& Culture},
  colorlinks=true,
  linkcolor={highlight},
  filecolor={Maroon},
  citecolor={Blue},
  urlcolor={highlight},
  pdfcreator={LaTeX via pandoc}}


\title{VM303-01 Studies in Digital Media \& Culture}
\author{}
\date{}
\begin{document}
\maketitle

\pagestyle{mystyle}


\begin{figure}

\begin{minipage}{0.49\linewidth}
\href{https://www.newyorker.com/culture/infinite-scroll/how-the-internet-turned-us-into-content-machines}{\pandocbounded{\includegraphics[keepaspectratio]{chayka_internet_final2-small.gif}}}\end{minipage}%
%
\begin{minipage}{0.02\linewidth}
\href{https://emerson.edu/academics/academic-departments/visual-media-arts}{Department
of Visual \& Media Arts}\\
\href{https://emerson.edu/}{Emerson College}\\
Spring Semester 2025\\
Tues/Thur 14 January---1 May 2024\\
6:00-7:45 p.m.\\
Ansin Building 604\\
\href{http://mroberts.emerson.build/}{Dr.~Martin Roberts}\\
\end{minipage}%
%
\begin{minipage}{0.49\linewidth}
Image source: Kyle Chayka,
``\href{https://www.newyorker.com/culture/infinite-scroll/how-the-internet-turned-us-into-content-machines}{How
the Internet Turned Us Into Content Machines}'' (\textbf{New Yorker}, 4
June 2022.\end{minipage}%

\end{figure}%

\subsection{Overview}\label{overview}

This course considers the nature and contemporary forms of digital
culture. Broadly speaking, this can be defined as the diverse range of
symbolic practices through which communities affirm and maintain their
cultural identities using digital media devices and interfaces in a
globally networked society. While these practices are structured by
deeply unequal power relations, are contradictory, and often come into
conflict with one another, collectively they constitute what may be
considered a global digital culture.

A key component of the course is the automation of various forms of
creative production, from writing to the visual arts, by
natural-language processing computational systems (generally referred to
as ``artificial intelligence'' or ``AI''). The course addresses some of
the many issues raised by such systems, with a particular focus on
questions of aesthetics and the increasingly contested relationship
between artists and algorithms. While such systems now demonstrably pass
the Turing Test (i.e.~pass as human or their products as
human-produced), they also compel us to reconsider what we mean by
``art,'' or ``intelligence'' itself.

\subsection{Format}\label{format}

This is primarily a critical-thinking course, although it includes a
practical and production component. This means that it encourages you to
think reflexively and analytically about the digitally-mediated cultural
practices that the course considers, as well as to participate in them;
for example, you will be invited to experiment with image-synthesis and
text-generating software and analyze the results using key concepts and
theoretical frameworks.

\subsection{Outcomes}\label{outcomes}

By the end of the course, students will:

\begin{itemize}
\tightlist
\item
  have acquired a deeper understanding of the social, cultural, and
  political dimensions of digital technologies and networked
  communication;
\item
  be able to apply critical thinking to contemporary developments in
  digital culture using relevant analytical concepts and both
  qualitative and quantitative methodologies such as cultural analytics;
\item
  understand basic principles of algorithmic image synthesis on a
  variety of platforms;
\item
  have reflected upon and discussed the larger significance of machine
  learning systems within global networked societies.
\end{itemize}

\subsection{Course Texts}\label{course-texts}

Selected chapters from the texts below will be made available as PDFs;
you are nevertheless encouraged to purchase at least several of texts
that are of interest and read more of them.

Note on formats: A number of texts listed in the
\href{bibliography.qmd}{Bibliography} are available as e-books and/or
audiobooks. You are encouraged to make use not only of print media but
also of these screen-based and audio formats.

\begin{itemize}
\tightlist
\item
  \href{https://wishcrys.com/about-me/}{Crystal Abidin},
  \textbf{Internet Celebrity: Understanding Fame Online} (Bingley, UK:
  Emerald Publishing, 2018).\\
\item
  \href{https://www.kylechayka.com/}{Kyle Chayka},
  \href{https://www.kylechayka.com/filterworld}{\textbf{Filterworld: How
  Algorithms Flattened Culture}}. New York: Doubleday, 2024.\\
\item
  Kazuo Ishiguro,
  \href{https://www.penguinrandomhouse.com/books/653825/klara-and-the-sun-a-gma-book-club-pick-by-kazuo-ishiguro/}{\textbf{Klara
  and the Sun: A Novel}}. New York: Penguin Random House, 2022.\\
\item
  \href{https://manovich.net/}{Lev Manovich} and Emanuele Arielli,
  \href{http://manovich.net/index.php/projects/artificial-aesthetics-book}{\textbf{Artificial
  Aesthetics: Generative AI, Art and Visual Media}}. 2019-24.\\
\item
  \href{https://joannemcneil.com/}{Joanne McNeil}.
  \href{https://us.macmillan.com/books/9781250785756/lurking/}{\textbf{Lurking:
  How A Person Became A User}} (New York: Farrar, Strauss, and Giroux,
  2020).\\
\item
  Liz Pelly,
  \href{https://www.simonandschuster.com/books/Mood-Machine/Liz-Pelly/9781668083505}{\textbf{Mood
  Machine: The Rise of Spotify and the Costs of the Perfect Playlist}}
  (New York: Simon \& Schuster, 2025).\\
\item
  Min-Ha T. Pham,
  \href{https://www.dukeupress.edu/asians-wear-clothes-on-the-internet}{\textbf{Asians
  Wear Clothes on the Internet: Race, Gender, and the Work of Personal
  Style Blogging}} (Durham, NC: Duke University Press, 2015).
\end{itemize}

\subsection{Substack: Recommended
blogs}\label{substack-recommended-blogs}

\href{https://www.freyaindia.co.uk/}{GIRLS} (Freya India)\\
\href{https://robhorning.substack.com/}{\emph{Internal Exile}} (Rob
Horning)\\
\href{https://substack.com/@jenka?utm_source=top-search}{Jenka
Gurfinkel}\\
\href{https://substack.com/@garymarcus?utm_source=top-search}{Marcus on
AI} (Gary Marcus)\\
\href{https://tellthebeees.substack.com/archive?sort=new}{Telling The
Bees}

\subsection{Other Sources}\label{other-sources}

\href{https://www.instagram.com/anthonyslook/}{Anthony's Look}
(\href{https://www.com.cuhk.edu.hk/people/fung-anthony-y-h/}{Anthony
Fung}, City University of Hong Kong)\\
\href{https://manovich.net}{Lev Manovich}\\
\href{https://www.instarbooks.com/books/everyword.html}{instar books}\\
\href{https://selfieresearchers.com}{Selfies Researchers Network}\\
\href{https://tiktokcultures.com/}{TikTok Cultures Research Network}

\subsection{Assignments \& Evaluation}\label{assignments-evaluation}

\textbf{Agenda (2) and Discussion Forum (20\%)}\\
Twice during the semester. Short response (max. 250 words) post to
weekly reading assignments, due on Canvas Tuesday at 12:00 pm. One
student will be assigned weekly, with the others required to post at
least one reply in the same week.

\textbf{Commentary (20\%)}\\
Weekly. Follow-up post (approx. 250 words) on assigned films or online
videos screened at Thursday class. Due by Sunday of the week in
question, and no later than the Tuesday of the week after.

\textbf{Aesthetics Case Study (15\%)}\\
Case study of an aesthetic movement or style, 1,000 words (4 pages,
double-spaced). Due mid-semester.

\textbf{Generative Art Project (15\%)}\\
Using one of the generative art platforms focused on in the course
(DALL-E 2, Midjourney, Stable Diffusion), submit one work that was
generated using one of these systems. Images may be still or moving
(e.g.~animations, GIF loops, etc.)

This work will be reviewed collectively by the group and displayed as a
gallery, initally on Canvas, and later (with your permission) on the
web.

\textbf{Research Paper/Project (20\%)}\\
Research on an approved topic relevant to the course. Individual or
group. Further details will be provided after Spring Break. 1,250-1,500
words.

\textbf{Engagement (10\%)}\\
Includes attendance, punctuality, preparation, participation in class
and/or online discussion,individual conferences.

\subsection{Schedule of Classes}\label{schedule-of-classes}

\emph{Week 1}

2025-01-14\_Tues

Introduction

2025-01-16\_Thur

\textbf{Smile for the Camera: Selfies}

\begin{itemize}
\tightlist
\item
  ``\href{https://www.reddit.com/r/midjourney/comments/11vuvdk/time_period_selfies_time_traveler_shows_soldiers/?rdt=64935}{Time
  Period Selfies}'' (\textbf{Reddit}, 2023)
\item
  Jenka Gurfinkel,
  ``\href{https://medium.com/@socialcreature/ai-and-the-american-smile-76d23a0fbfaf}{AI
  and the American Smile}'' (\textbf{Medium}, 17 March 2023)
\end{itemize}

\emph{Week 2}

Topic: \textbf{Virtual Communities}

2025-01-21\_Tues

\href{https://joannemcneil.com/}{Joanne McNeil},
\href{https://us.macmillan.com/books/9781250785756/lurking/}{\textbf{Lurking:
How A Person Became A User}}:

\begin{itemize}
\tightlist
\item
  \href{./pdf/Lurking_Intro.pdf}{Introduction}
\item
  \href{./pdf/Lurking_Chapter2.pdf}{Chapter 2}
\end{itemize}

2025-01-23\_Thur

\emph{Week 3}

Topic: \textbf{Lifestyle}

2025-01-28\_Tues

Min-Ha T. Pham,
\href{https://www.dukeupress.edu/asians-wear-clothes-on-the-internet}{\textbf{Asians
Wear Clothes on the Internet: Race, Gender, and the Work of Personal
Style Blogging}}, selected chs.

2025-01-30\_Thur

\href{https://www.instagram.com/anthonyslook/}{Anthony's Look} and other
fashion blogs tba

\emph{Week 4}

Topic: \textbf{Creators}

2025-02-04\_Tues

\begin{itemize}
\tightlist
\item
  Stuart Cunningham and David Craig, \textbf{Creator Culture: An
  Introduction to Global Social Media Entertainment},
  \href{pdf/creator-culture-toc-intro.pdf}{Introduction}
\item
  Elaine Jing Zhao, ``\emph{Wanghong}: Liminial Chinese Creative Labor''
  (\textbf{Creator Culture}, ch.~11)
\item
  Roland Kelts, ``\href{https://restofworld.org/2021/vtubers/}{Japan's
  virtual YouTubers have millions of real subscribers --- and make
  millions of real dollars}'' (\textbf{Rest of World}, 26 July 2021)
\end{itemize}

2025-02-06\_Thur

\emph{Week 5}

Topic: \textbf{Aesthetics}

2025-02-11\_Tues

\begin{itemize}
\tightlist
\item
  ``\href{https://aesthetics.fandom.com/wiki/Aesthetics_Wiki}{What Are
  Aesthetics?}'' (\textbf{Aesthetics Wiki}) (read the articles in this
  section)
\item
  Guilherme Giolo and Michaël Berghman,
  ``\href{pdf/giolo-berghman-aesthetics.pdf}{The aesthetics of the self:
  The meaning-making of Internet aesthetics}
\end{itemize}

2025-02-13\_Thur

\emph{Week 6}

2025-02-18\_Tues NO CLASS (Mon schedule)

\textbf{Coded Bias} (documentary, available on Netflix)

2025-02-20\_Thur

Topic: \textbf{Algorithm}

\href{https://www.kylechayka.com/}{Kyle Chayka},
\href{https://www.kylechayka.com/filterworld}{\textbf{Filterworld}}
(selected chs.)

\textbf{DEADLINE: Aesthetics case study}

\emph{Week 7}

Topic: \textbf{Influencers and Celebrities}

2025-02-25\_Tues

Crystal Abidin, \textbf{Internet Celebrities} (selected chs.)

2025-02-27\_Thur

\emph{Week 8}

Topic: \textbf{Virtual Worlds}

2025-03-04\_Tues

Screening: \textbf{We Met In Virtual Reality}\\
Readings tba

2025-05-06\_Thur

\textbf{SPRING BREAK}

\emph{Week 9}

Topic: \textbf{Chatbots}

2025-03-18\_Tues

ELIZA Replika

2025-03-20\_Thur

\textbf{Her} (Spike Jonze, 2013)

\emph{Week 10}

Topic: \textbf{Artificial Friends}

2025-03-25\_Tues

Kazuo Ishiguro, \textbf{Klara and the Sun}

2025-03-27\_Thur

Film: \textbf{After Yang} (kogonada, 2021)

\emph{Week 11}

Topic: \textbf{Generative Media}

2025-04-01\_Tues

\href{http://www.generative-gestaltung.de/2/}{Generative Design}
(website)

\href{https://www.generativehut.com/}{Generative Hut} (website)

2025-04-03\_Thur

Workshop: Introduction to P5.js

\emph{Week 12}

Topic: \textbf{Algorithmic Aesthetics}

2025-04-08\_Tues

Lev Manovich and Emmanuele Arielli,
\href{https://manovich.net/index.php/projects/artificial-aesthetics}{\textbf{Artificial
Aesthetics}}:

\begin{itemize}
\tightlist
\item
  ``Who is an Artist in AI Era?'' (ch.~2)
\end{itemize}

2025-04-10\_Thur

\begin{itemize}
\tightlist
\item
  ``AI and Myths of Creativity'' (ch.~4)
\end{itemize}

Workshop: Introduction to Midjourney

\emph{Week 13}

Topic: \textbf{Algorithmic Aesthetics} (cont.)

2025-04-15\_Tues

Lev Manovich and Emmanuele Arielli,
\href{https://manovich.net/index.php/projects/artificial-aesthetics}{\textbf{Artificial
Aesthetics}}:

\begin{itemize}
\tightlist
\item
  ``Seven Arguments About AI Images and Generative Media'' (ch.~5)
\end{itemize}

Workshop: Introduction to ComfyUI

2025-04-17\_Thur NO CLASS (Make-up Day)

\emph{Week 14}

2025-04-24\_Tues

Topic: \textbf{Algorithmic Aesthetics} (cont.)

Lev Manovich and Emmanuele Arielli,
\href{https://manovich.net/index.php/projects/artificial-aesthetics}{\textbf{Artificial
Aesthetics}}:

\begin{itemize}
\tightlist
\item
  ``Separate and Reassemble: Generative AI and Media History'' (ch.~7)
\end{itemize}

2025-04-25\_Thur

Generative Art Projects

\emph{Week 15}

2025-04-29\_Tues

Presentations: Research Paper/Project

2025-05-01\_Thur

Presentations: Research Paper/Project

2025-05-01 Fri \textbf{Last day of classes}

\subsection{Policies}\label{policies}

\textbf{Academic Honesty}

It is the responsibility of all Emerson students to know and adhere to
the College's policy on plagiarism, which can be found
at~\href{https://emerson.edu/policies/plagiarism}{emerson.edu/policies/plagiarism}.
If you have any question concerning the Emerson plagiarism policy or
about documentation of sources in work you produce in this course, speak
to your instructor.

\textbf{Diversity}

Every student in this class will be honored and respected as an
individual with distinct experiences, talents, and backgrounds. Issues
of diversity may be a part of class discussion, assigned material, and
projects. The instructor will make every effort to ensure that an
inclusive environment exists for all students. If you have any concerns
or suggestions for improving the classroom climate, please do not
hesitate to speak with the course instructor or to contact the Social
Justice Center at 617-824-8528 or by email
at~\href{mailto:sjc@emerson.edu}{\nolinkurl{sjc@emerson.edu}}.

\textbf{Discrimination, Harassment, or Sexual Violence}

If you have been impacted by discrimination, harassment, or sexual
violence, I am available to support you, and help direct you to
available resources on and off campus. Additionally, the Office of Equal
Opportunity (\href{mailto:oeo@emerson.edu}{\nolinkurl{oeo@emerson.edu}};
617-824-8999) is available to meet with you and discuss options to
address concerns and to provide you with support resources. Please note
that I because I am an Emerson employee, any information shared with me
related to discrimination, harassment, or sexual violence will also be
shared with the Office of Equal Opportunity. ~If you would like to speak
with someone confidentially, please contact the Healing \& Advocacy
Collective, the Emerson Wellness Center, or the Center for Spiritual
Life.

\textbf{Accessibility}

Emerson is committed to providing equal access and support to all
students who qualify through the provision of reasonable accommodations,
so that each student may fully participate in the Emerson experience. If
you have a disability that may require accommodations, please contact
Student Accessibility Services (SAS)
at~\href{mailto:SAS@emerson.edu}{\nolinkurl{SAS@emerson.edu}}~or
617-824-8592 to make an appointment with an SAS staff member.

Students are encouraged to contact SAS early in the semester. Please be
aware that accommodations are not applied retroactively.

\textbf{Writing \& Academic Resource Center}

Students are encouraged to visit and utilize the staff and resources of
Emerson's Writing Center, particularly if they are struggling with
written assignments. The Writing Center is located at 216 Tremont Street
on the 5th floor (tel. 617-824-7874).

\textbf{In-Class Recording}

Regardless of modality or whether this course is being recorded by the
College with the permission of the students for classroom purposes, this
class is considered a private environment and it is a setting in which
copyrighted materials, creative works and educational records may be
displayed. Audio or video recording, filming, photographing, viewing,
transmitting, producing or publishing the image or voice of another
person or that person's materials, creative works or educational records
without the person's knowledge and expressed consent is strictly
prohibited.~

\begin{center}\rule{0.5\linewidth}{0.5pt}\end{center}




\end{document}
